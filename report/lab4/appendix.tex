\appendix

\section{附录:源代码(带注释)}

\begin{verbatim}
    import numpy as np
    import matplotlib.pyplot as plt
    import PIL.Image as pm


    def vectors2xylist(vectors: list) -> tuple:
        """转换向量列表格式
        将
        [
            [x0, y0],
            [x1, y1],
            ...,
            [xn, yn],
        ]
        格式的 vectors 转换为
        [x0, x1, ... xn], [y0, y1, ... yn]
        两个列表

        Args:
            vectors:  待转换向量列表

        Returns:
            返回向量列表中各组向量的 x 坐标和 y 坐标
        """

        xs = []
        ys = []
        for vec in vectors:
            xs.append(vec[0])
            ys.append(vec[1])

        return xs, ys


    def center(data: list, data_count: int) -> tuple:
        """对数据列表进行均值处理:
        对每一个数据,都减去他们的均值向量,成为新的数据向量

        Args:
            data:       数据集列表
            data_count: 数据数量

        Returns:
            返回 (处理后的数据集, 原数据集的均值向量)
        """

        # 训练样本向量均值
        data_center: list = np.mean(data, axis=0)

        centered_data = []
        for d in data:
            centered_data.append(d - data_center)

        return centered_data, data_center


    def pca(data: list, data_count: int, dim: int, dim_down: int) -> np.array:
        """使用 PCA 算法进行降维

        Args:
            data:       训练数据
            data_count: 数据量
            dim:        数据向量的维度
            dim_down:   降维后的维度

        Returns:
            返回按照特征值从大到小排序的特征向量矩阵,矩阵中的特征向量是列向量
        """

        centered_data, _ = center(data, data_count)

        # 样本协方差矩阵
        cov = np.zeros((dim, dim))
        for i in range(0, data_count):
            cov += np.dot(
                centered_data[i].reshape((-1, 1)),
                np.transpose(centered_data[i]).reshape((1, -1)),
            )
        cov /= data_count

        # 求特征向量和特征值
        lams, vectors = np.linalg.eig(cov)  # vectors 是列向量

        # 根据特征值对特征向量进行排序
        index = np.argsort(lams)  # 从小到大排序后的下标序列
        vectors = vectors[:, index[: -(dim_down + 1) : -1]]  # 把序列逆向排列然后取前k个

        return vectors


    def dim2(data_count: int, mu: list, sigma: list) -> None:
        """使用 PCA 算法对二维数据进行降维

        Args:
            data_count: 样本数据数量
            mu:         样本数据均值
            sigma:      样本数据协方差矩阵
        """

        # ===== 初始参数 =====
        data: list = []
        dim: int = 2

        # ===== 生成训练数据 =====
        for i in range(0, data_count):
            data.append(np.random.multivariate_normal(mu, sigma))
        xs, ys = vectors2xylist(data)
        plt.scatter(xs, ys, label="origin points")

        # ===== 降维 =====
        vectors = pca(data, data_count, dim, 1 + 1)
        xs = [i for i in range(-2, 3)]
        ys = []
        for x in range(-2, 3):
            ys.append(vectors[1][0] / vectors[1][1] * x)
        plt.plot(xs, ys, c="g", label="low dim line")

        # 画出降维后的点
        low_data = []
        vec = vectors[:1]
        for i in data:
            low_data.append(np.dot(i, vectors.T[0]) * vectors.T[0])
        xs, ys = vectors2xylist(low_data)
        plt.scatter(xs, ys, c="r", label="low dim points")

        # ===== 旋转 =====
        vectors = pca(data, data_count, dim, dim)
        xs = [i for i in range(0, 2)]
        ys = []
        for x in range(0, 2):
            ys.append(vectors[0][0] / vectors[0][1] * x)
        plt.plot(xs, ys, c="b", label="new axis 0")
        xs = [i for i in range(0, 2)]
        ys = []
        for x in range(0, 2):
            ys.append(vectors[1][0] / vectors[1][1] * x)
        plt.plot(xs, ys, c="b", label="new axis 1")

        # ===== 绘图 =====
        plt.axis("scaled")  # 绘制正方形,长宽比例相同


    def face(img_path: str, low_dim: int) -> None:
        """使用 PCA 算法对人脸数据进行降维

        Args:
            img_path: 图片文件路径
            low_dim:  降维到的维度
        """

        # ===== 读取文件 =====
        face = np.array(pm.open(img_path).convert("L"))  # 灰度图

        # ===== PCA =====
        vectors = pca(face, face.shape[0], face.shape[1], low_dim)
        centered_data, data_center = center(face, face.shape[0])

        # ===== 重建数据 =====
        pca_face = (
            np.dot((np.dot(np.array(centered_data), vectors)), vectors.T) + data_center
        )

        # ===== 信噪比 =====
        noise = np.sqrt(np.mean((face - pca_face) ** 2))
        snr = 20 * np.log10(255 / noise)

        # ===== 绘图 =====
        plt.imshow(pca_face)
        plt.title("dim = " + str(low_dim) + "\nSNR = " + str(snr))


    if __name__ == "__main__":
        dim2(100, mu=[0, 0], sigma=[[10, 3], [3, 1]])
        # face("lab4/img/2.jpg", 10)

        # ===== 绘图 =====
        plt.legend()
        plt.show()
\end{verbatim}
