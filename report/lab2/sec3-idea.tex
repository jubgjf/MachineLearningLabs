\section{设计思想(本程序中的用到的主要算法及数据结构)}

\subsection{逻辑回归算法原理}

\subsubsection{似然函数}

对于一个二分类问题,有类别 $Y \in \{ 0, 1 \}$,以及样本数据向量
\begin{equation}
    \mathbf{X} =
    \begin{bmatrix}
        x_1 \\ x_2 \\ \vdots \\ x_n
    \end{bmatrix}
\end{equation}

$P(x_i | Y = y_k)$相互独立,且$P(x_i | Y = y_k) \sim \mathcal{N} (\mu_{ik}, \sigma_i)$。
同时$P(Y) \sim \mathcal{B} (\pi)$。设$P(Y = 0) = \pi$,$p(Y = 1) = 1 - \pi$。

对于一个新样本$\mathbf{X}$,设
\begin{equation}
    \mathbf{\hat{X}}=
    \begin{bmatrix}
        1 \\ \mathbf{X}
    \end{bmatrix}=
    \begin{bmatrix}
        1 \\ x_1 \\ \vdots \\ x_n
    \end{bmatrix}
\end{equation}
为了便于后续化简,以后均用$\mathbf{X}$代表$\mathbf{\hat{X}}$,即
\begin{equation}
    \mathbf{X}=
    \begin{bmatrix}
        x_0=1 \\ x_1 \\ \vdots \\ x_n
    \end{bmatrix}
\end{equation}

设$\mathbf{X}$属于类别$0$的概率为$P(Y=0|\mathbf{X})$,则有
\begin{align}
    \label{equ1}
    P(Y=0|\mathbf{X})   &=  \dfrac{P(\mathbf{X}|Y = 0) P(Y = 0)}{P(\mathbf{X}|Y = 0) P(Y = 0) + P(\mathbf{X}|Y=1) P(Y=1)} \\
                        &=  \dfrac{1}{1+\dfrac{P(\mathbf{X}|Y=1) P(Y=1)}{P(\mathbf{X}|Y=0) P(Y=0)}} \\
                        &=  \dfrac{1}{1+\exp\left(\ln \dfrac{P(\mathbf{X}|Y=1) P(Y=1)}{P(\mathbf{X}|Y=0) P(Y=0)}\right)} \\
                        &=  \dfrac{1}{1+\exp\left(\ln \dfrac{P(\mathbf{X}|Y=1)}{P(\mathbf{X}|Y=0)} +\ln \dfrac{1 - \pi}{\pi}\right)} \\
                        &=  \dfrac{1}{1+\exp\left(\ln\left(\dfrac{\prod\limits^n_{i=0} P(x_i|Y=1)}{\prod\limits^n_{i=0} P(x_i|Y=0)}\right) +\ln \dfrac{1 - \pi}{\pi}\right)} \\
                        &=  \dfrac{1}{1+\exp\left(\sum\limits^n_{i=0}\ln\left(\dfrac{P(x_i|Y=1)}{P(x_i|Y=0)}\right) +\ln \dfrac{1 - \pi}{\pi}\right)} \\
                        &=  \dfrac{1}{1+\exp\left(\sum\limits^n_{i=1}\ln\left(\dfrac{\sqrt{2\pi}\sigma_i  \exp\left(-\dfrac{\left(x_i-\mu_{i1}\right)^2}{2\sigma^2_i}\right)}{\sqrt{2\pi}\sigma_i \exp\left(-\dfrac{\left(x_i-\mu_{i0}\right)^2}{2\sigma^2_i}\right)}\right)+\ln \dfrac{1-\pi}{\pi}\right)} \\
                        &=  \dfrac{1}{1+\exp\left(\sum\limits^n_{i=1}\left(-\dfrac{\left(x_i-\mu_{i1}\right)^2}{2\sigma^2_i}+\dfrac{\left(x_i-\mu_{i0}\right)^2}{2\sigma^2_i}\right)+\ln \dfrac{1-\pi}{\pi}\right)} \\
                        &=  \dfrac{1}{1+\exp\left(\sum\limits^n_{i=1}\left(\dfrac{\mu_{i1}-\mu_{i0}}{\sigma^2_i}x_i\right)+\sum\limits^n_{i=1}\dfrac{\mu^2_{i1}-\mu^2_{i2}}{2\sigma^2_i} +\ln\dfrac{1-\pi}{\pi}\right)} \\
                        &=  \dfrac{1}{1+\exp\left(\mathbf{w}^T\mathbf{X}\right)}
\end{align}
其中
\begin{equation}
    \mathbf{w} =
    \begin{bmatrix}
        \sum\limits^n_{i=1}\dfrac{\mu^2_{i1}-\mu^2_{i2}}{2\sigma^2_i} +\ln\dfrac{1-\pi}{\pi} \\
        \dfrac{\mu_{11}-\mu_{10}}{\sigma^2_1} \\
        \vdots \\
        \dfrac{\mu_{i1}-\mu_{i0}}{\sigma^2_i} \\
        \vdots \\
        \dfrac{\mu_{n1}-\mu_{n0}}{\sigma^2_n} \\
    \end{bmatrix}
\end{equation}
由于$P(Y=1|\mathbf{X})+P(Y=0|\mathbf{X})=1$,所以
\begin{equation}
    P(Y=1|\mathbf{X})=\dfrac{\exp\left(\mathbf{w}^T \mathbf{X}\right)}{1+\exp\left(\mathbf{w}^T \mathbf{X}\right)}
\end{equation}

由此可得分类的原理:对于一个新样本$\mathbf{X}$,分别计算其属于$0$类和$1$类的概率,取概率大者作为$\mathbf{X}$的类别。

由
\begin{equation}
    \label{sep1}
    \dfrac{P(Y=0|\mathbf{X})}{P(Y=1|\mathbf{X})} = \exp\left(\mathbf{w}^T\mathbf{X}\right)
\end{equation}
两边同时取对数,可得
\begin{equation}
    \label{sep2}
    \ln\dfrac{P(Y=0|\mathbf{X})}{P(Y=1|\mathbf{X})} = \mathbf{w}^T\mathbf{X}
\end{equation}
则当式\ref{sep2}大于$0$,即式\ref{sep1}大于$1$时,$\mathbf{X}$的类别为$0$,否则为类别$1$。故分类面为
\begin{equation}
    \mathbf{w}^T\mathbf{X}=0
\end{equation}

对于$L$组数据$\{\langle\mathbf{X}^1,Y^1\rangle,\ldots,\langle\mathbf{X}^l,Y^l\rangle,\ldots,\langle\mathbf{X}^L,Y^L\rangle\}$,其中$Y^l$是样本数据向量$\mathbf{X}^l$的类别。使用条件最大似然估计,对参数$\mathbf{w}$进行分析
\begin{equation}
    \mathbf{w}_{MCLE}=arg\max_{\mathbf{w}}\prod^L_{l=1}P\left(Y^l|\mathbf{X}^l,\mathbf{w}\right)
\end{equation}
则条件似然函数的对数为
\begin{align}
    \label{likelihood}
    l(\mathbf{w})
    &= \ln\left(\prod^L_{l=1}P\left(Y^l|\mathbf{X}^l,\mathbf{w}\right)\right) \\
    &= \sum^L_{l=1}\ln\left(P\left(Y^l|\mathbf{X}^l,\mathbf{w}\right)\right) \\
    &= \sum^L_{l=1}\left(Y^l\ln\left(P\left(Y^l=1|\mathbf{X}^l,\mathbf{w}\right)\right)+(1-Y^l)\ln\left(P\left(Y^l=0|\mathbf{X}^l,\mathbf{w}\right)\right)\right) \\
    &= \sum^L_{l=1}\left(Y^l\ln\left(\dfrac{P\left(Y^l=1|\mathbf{X}^l,\mathbf{w}\right)}{P\left(Y^l=0|\mathbf{X}^l,\mathbf{w}\right)}\right)+\ln\left(P\left(Y^l=0|\mathbf{X}^l,\mathbf{w}\right)\right)\right) \\
    \label{exp1}
    &= \sum^L_{l=1}\left(Y^l\left(\mathbf{w}^T\mathbf{X}^l\right)-\ln\left(1+\exp\left(\mathbf{w}^T\mathbf{X}^l\right)\right)\right) \\
    &= \sum^L_{l=1}\left(Y^l\sum^n_{i=1}w_ix^l_i-\ln\left(1+\exp\sum^n_{i=1}w_ix^l_i\right)\right) \\
    &= \sum^L_{l=1}\left(Y^l\mathbf{w}^T\mathbf{X}^l-\ln\left(1+\exp\left(\mathbf{w}^T\mathbf{X}^l\right)\right)\right)
\end{align}

要最大化式\ref{likelihood}中的对数条件似然函数,只需最小化其相反数,即
\begin{equation}
    \mathcal{L} (\mathbf{w})=\sum^L_{l=1}\left(-Y^l\mathbf{w}^T\mathbf{X}^l+\ln\left(1+\exp\left(\mathbf{w}^T\mathbf{X}^l\right)\right)\right)
\end{equation}

为了降低模型的复杂度,仿照实验一中的做法,在对数条件似然函数中补充一个正则项(惩罚项),即
\begin{equation}
    \mathcal{L} (\mathbf{w})=\sum^L_{l=1}\left(-Y^l\mathbf{w}^T\mathbf{X}^l+\ln\left(1+\exp\left(\mathbf{w}^T\mathbf{X}^l\right)\right)\right)+\dfrac{\lambda}{2}\left\lVert \mathbf{w}\right\rVert ^2
\end{equation}
其中
\begin{equation}
    \left\lVert \mathbf{w}\right\rVert ^2 
\end{equation}
为$\mathbf{w}$的二范数,即
\begin{equation}
    \left\lVert \mathbf{w}\right\rVert ^2=\mathbf{w}^T\mathbf{w}
\end{equation}

\subsubsection{凸优化}

Longlongaaago\cite{tu}利用Hessian矩阵,证明了逻辑回归的目标函数$\mathcal{L} (\mathbf{w})$为凸函数,因此我们可以用梯度下降法、牛顿迭代法等方式找到其极小值。

\subsubsection{梯度下降法}

梯度下降迭代公式为
\begin{align}
    \mathbf{w}_{k+1}
    &= \mathbf{w}_k-\alpha\dfrac{\partial\mathcal{L}(\mathbf{w})}{\partial\mathbf{w}} \\
    &= \mathbf{w}_k-\alpha\sum^L_{l=1}\left(-Y^l\mathbf{X}^l+\dfrac{\exp\left(\mathbf{w}^T\mathbf{X}^l\right)}{1+\exp\left(\mathbf{w}^T\mathbf{X}^l\right)}\mathbf{X}^l\right) \\
    &= \mathbf{w}_k+\alpha\sum^L_{l=1}\mathbf{X}^l\left(Y^l-\dfrac{\exp\left(\mathbf{w}^T\mathbf{X}^l\right)}{1+\exp\left(\mathbf{w}^T\mathbf{X}^l\right)}\right)
\end{align}

在添加正则项之后,利用向量求导公式\cite{dvec},可得迭代公式为
\begin{equation}
    \label{gd_final}
    \mathbf{w}_{k+1}=\mathbf{w}_k+\alpha\left(\sum^L_{l=1}\mathbf{X}^l\left(Y^l-\dfrac{\exp\left(\mathbf{w}^T\mathbf{X}^l\right)}{1+\exp\left(\mathbf{w}^T\mathbf{X}^l\right)}\right)-\lambda\mathbf{w}\right)
\end{equation}

\subsubsection{牛顿迭代法}

牛顿法的几何意义为:在每一次迭代中,均以一个二次函数去逼近$\mathcal{L}(x)$。对于高维问题,牛顿法的迭代公式可见式\ref{nt}
\begin{equation}
    \label{nt}
    \mathbf{w}_{k+1}=\mathbf{w}_k-\left(\mathbf{H}\mathcal{L}(\mathbf{w})\right)^{-1}\nabla \mathcal{L}(\mathbf{w})
\end{equation}
其中$\mathbf{H}\mathcal{L}(\mathbf{w})$为海森矩阵,在此问题中相当于求二阶导,即
\begin{equation}
    \mathbf{w}_{k+1}=\mathbf{w}_k-\left(\nabla^2\mathcal{L}(\mathbf{w})\right)^{-1}\nabla \mathcal{L}(\mathbf{w})
\end{equation}
因此牛顿法关键是计算$\nabla \mathcal{L}(\mathbf{w})$和$\nabla^2\mathcal{L}(\mathbf{w})$。

对于$\nabla \mathcal{L}(\mathbf{w})$,有
\begin{align}
    \nabla\mathcal{L}(\mathbf{w})
    &= \dfrac{\partial\mathcal{L}(\mathbf{w})}{\partial\mathbf{w}} \\
    &= \sum^L_{l=1}\mathbf{X}^l\left(-Y^l+\dfrac{\exp\left(\mathbf{w}^T\mathbf{X}^l\right)}{1+\exp\left(\mathbf{w}^T\mathbf{X}^l\right)}\right)
\end{align}

对于$\nabla^2\mathcal{L}(\mathbf{w})$,有
\begin{align}
    \nabla^2\mathcal{L}(\mathbf{w})
    &= \dfrac{\partial^2\mathcal{L}(\mathbf{w})}{\partial\mathbf{w}\partial\mathbf{w}^T} \\
    &= \dfrac{\partial\left(\sum\limits^L_{l=1}\mathbf{X}^l\left(-Y^l+\dfrac{\exp\left(\mathbf{w}^T\mathbf{X}^l\right)}{1+\exp\left(\mathbf{w}^T\mathbf{X}^l\right)}\right)\right)}{\partial\mathbf{w}^T} \\
    &= \sum^L_{l=1}\mathbf{X}^l\dfrac{\exp\left(\mathbf{w}^T\mathbf{X}^l\right){\mathbf{X}^l}^T}{\left(1+\exp\left(\mathbf{w}^T\mathbf{X}^l\right)\right)^2}
\end{align}
所以牛顿法的迭代公式为
\begin{equation}
    \mathbf{w}_{k+1}=\mathbf{w}_k-\left(\sum^L_{l=1}\mathbf{X}^l{\mathbf{X}^l}^T\dfrac{\exp\left(\mathbf{w}^T\mathbf{X}^l\right)}{\left(1+\exp\left(\mathbf{w}^T\mathbf{X}^l\right)\right)^2}\right)^{-1}\left(\sum^L_{l=1}\mathbf{X}^l\left(-Y^l+\dfrac{\exp\left(\mathbf{w}^T\mathbf{X}^l\right)}{1+\exp\left(\mathbf{w}^T\mathbf{X}^l\right)}\right)\right)
\end{equation}

在添加正则项之后,迭代公式为
\begin{equation}
    \mathbf{w}_{k+1}=\mathbf{w}_k-\left(\sum^L_{l=1}\mathbf{X}^l{\mathbf{X}^l}^T\dfrac{\exp\left(\mathbf{w}^T\mathbf{X}^l\right)}{\left(1+\exp\left(\mathbf{w}^T\mathbf{X}^l\right)\right)^2}-\lambda\right)^{-1}\left(\sum^L_{l=1}\mathbf{X}^l\left(-Y^l+\dfrac{\exp\left(\mathbf{w}^T\mathbf{X}^l\right)}{1+\exp\left(\mathbf{w}^T\mathbf{X}^l\right)}\right)-\lambda\mathbf{w}\right)
\end{equation}
