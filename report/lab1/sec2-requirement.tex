\section{实验要求及实验环境}

\subsection{实验要求}

1. 生成数据,加入噪声;

2. 用高阶多项式函数拟合曲线;

3. 用解析解求解两种loss的最优解(无正则项和有正则项)

4. 优化方法求解最优解(梯度下降,共轭梯度);

5. 用你得到的实验数据,解释过拟合。

6. 用不同数据量,不同超参数,不同的多项式阶数,比较实验效果。

7. 语言不限,可以用matlab,python。求解解析解时可以利用现成的矩阵求逆。梯度下降,共轭梯度要求自己求梯度,迭代优化自己写。不许用现成的平台,例如pytorch,tensorflow的自动微分工具。

\subsection{实验环境}

Python(Numpy + Matplotlib)
