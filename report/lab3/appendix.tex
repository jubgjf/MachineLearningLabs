\appendix

\section{附录:多元正态分布的求导}

多元正态分布定义为
\begin{equation}
    \mathcal{N}\left(\mathbf{x}|\mu,\Sigma\right)=
    \dfrac{1}{\left(2\pi\right)^{\frac{d}{2}}|\Sigma|^{\frac{1}{2}}}\exp\left(-\dfrac{1}{2}\left(\mathbf{x}-\mu\right)^T\Sigma^{-1}\left(\mathbf{x}-\mu\right)\right)
\end{equation}
令其对$\mu$求导(用到的向量求导公式见文献\cite{cookbook})
\begin{align}
    \dfrac{\partial\mathcal{N}\left(\mathbf{x}|\mu,\Sigma\right)}{\partial\mu}
    &= \dfrac{\partial}{\partial\mu}\left(\dfrac{1}{\left(2\pi\right)^{\frac{d}{2}}|\Sigma|^{\frac{1}{2}}}\exp\left(-\dfrac{1}{2}\left(\mathbf{x}-\mu\right)^T\Sigma^{-1}\left(\mathbf{x}-\mu\right)\right)\right) \\
    &= \dfrac{1}{\left(2\pi\right)^{\frac{d}{2}}|\Sigma|^{\frac{1}{2}}}\exp\left(-\dfrac{1}{2}\left(\mathbf{x}-\mu\right)^T\Sigma^{-1}\left(\mathbf{x}-\mu\right)\right)\left(-\dfrac{1}{2}\right)\left(\Sigma^{-1}+\Sigma^{-T}\right)\left(\mathbf{x}-\mu\right) \\
    &= \mathcal{N}\left(\mathbf{x}|\mu,\Sigma\right)\left(-\dfrac{1}{2}\right)\left(\Sigma^{-1}+\Sigma^{-T}\right)\left(\mathbf{x}-\mu\right)
\end{align}
假设$\mathbf{x}$的各个维度相互独立,即$\mathbf{x}_i$与$\mathbf{x}_j$相关系数为$0$($i\neq j$),则此时$\Sigma$为对称阵,则仍可进一步化简为
\begin{equation}
    \dfrac{\partial\mathcal{N}\left(\mathbf{x}|\mu,\Sigma\right)}{\partial\mu}
    =-\mathcal{N}\left(\mathbf{x}|\mu,\Sigma\right)\Sigma^{-1}\left(\mathbf{x}-\mu\right)
\end{equation}

\section{附录:Iris数据集训练过程}

\begin{verbatim}
    [....................][GMM-EM] ln(likehood) = -792.7784356553897
    [....................][GMM-EM] ln(likehood) = -724.765219839081
    [....................][GMM-EM] ln(likehood) = -722.0911759517809
    [....................][GMM-EM] ln(likehood) = -721.8616486642446
    [#...................][GMM-EM] ln(likehood) = -721.7732146022555
    [#...................][GMM-EM] ln(likehood) = -721.7275099321231
    [#...................][GMM-EM] ln(likehood) = -721.7015111255241
    [#...................][GMM-EM] ln(likehood) = -721.6857406017493
    [#...................][GMM-EM] ln(likehood) = -721.6756869937758
    [##..................][GMM-EM] ln(likehood) = -721.6690110008848
    [##..................][GMM-EM] ln(likehood) = -721.6644192149269
    [##..................][GMM-EM] ln(likehood) = -721.6611591168632
    [##..................][GMM-EM] ln(likehood) = -721.658774347985
    [##..................][GMM-EM] ln(likehood) = -721.6569785073738
    [###.................][GMM-EM] ln(likehood) = -721.6555866339503
    [###.................][GMM-EM] ln(likehood) = -721.6544763568434
    [###.................][GMM-EM] ln(likehood) = -721.6535650362042
    [###.................][GMM-EM] ln(likehood) = -721.6527958761055
    [###.................][GMM-EM] ln(likehood) = -721.6521292504578
    [####................][GMM-EM] ln(likehood) = -721.6515371517619
    [####................][GMM-EM] ln(likehood) = -721.6509995617004
    [####................][GMM-EM] ln(likehood) = -721.6505020330358
    [####................][GMM-EM] ln(likehood) = -721.650034051401
    [####................][GMM-EM] ln(likehood) = -721.6495879088853
    [#####...............][GMM-EM] ln(likehood) = -721.6491579192499
    [#####...............][GMM-EM] ln(likehood) = -721.6487398647342
    [#####...............][GMM-EM] ln(likehood) = -721.6483306020205
    [#####...............][GMM-EM] ln(likehood) = -721.6479277789616
    [#####...............][GMM-EM] ln(likehood) = -721.6475296292593
    [######..............][GMM-EM] ln(likehood) = -721.6471348225542
    [######..............][GMM-EM] ln(likehood) = -721.6467423542591
    [######..............][GMM-EM] ln(likehood) = -721.6463514641421
    [######..............][GMM-EM] ln(likehood) = -721.645961575855
    [######..............][GMM-EM] ln(likehood) = -721.6455722518064
    [#######.............][GMM-EM] ln(likehood) = -721.645183159371
    [#######.............][GMM-EM] ln(likehood) = -721.6447940454893
    [#######.............][GMM-EM] ln(likehood) = -721.6444047175197
    [#######.............][GMM-EM] ln(likehood) = -721.6440150287693
    [#######.............][GMM-EM] ln(likehood) = -721.6436248675325
    [########............][GMM-EM] ln(likehood) = -721.6432341487707
    [########............][GMM-EM] ln(likehood) = -721.6428428077907
    [########............][GMM-EM] ln(likehood) = -721.6424507954268
    [########............][GMM-EM] ln(likehood) = -721.6420580743744
    [########............][GMM-EM] ln(likehood) = -721.6416646163855
    [#########...........][GMM-EM] ln(likehood) = -721.6412704001361
    [#########...........][GMM-EM] ln(likehood) = -721.6408754095883
    [#########...........][GMM-EM] ln(likehood) = -721.640479632741
    [#########...........][GMM-EM] ln(likehood) = -721.6400830606774
    [#########...........][GMM-EM] ln(likehood) = -721.639685686835
    [##########..........][GMM-EM] ln(likehood) = -721.639287506441
    [##########..........][GMM-EM] ln(likehood) = -721.6388885160919
    [##########..........][GMM-EM] ln(likehood) = -721.6384887134207
    [##########..........][GMM-EM] ln(likehood) = -721.6380880968496
    [##########..........][GMM-EM] ln(likehood) = -721.6376866653985
    [###########.........][GMM-EM] ln(likehood) = -721.6372844185394
    [###########.........][GMM-EM] ln(likehood) = -721.6368813560813
    [###########.........][GMM-EM] ln(likehood) = -721.6364774780898
    [###########.........][GMM-EM] ln(likehood) = -721.6360727848191
    [###########.........][GMM-EM] ln(likehood) = -721.6356672766651
    [############........][GMM-EM] ln(likehood) = -721.6352609541241
    [############........][GMM-EM] ln(likehood) = -721.6348538177666
    [############........][GMM-EM] ln(likehood) = -721.6344458682161
    [############........][GMM-EM] ln(likehood) = -721.6340371061309
    [############........][GMM-EM] ln(likehood) = -721.6336275321938
    [#############.......][GMM-EM] ln(likehood) = -721.6332171471014
    [#############.......][GMM-EM] ln(likehood) = -721.6328059515547
    [#############.......][GMM-EM] ln(likehood) = -721.6323939462601
    [#############.......][GMM-EM] ln(likehood) = -721.631981131919
    [#############.......][GMM-EM] ln(likehood) = -721.6315675092284
    [##############......][GMM-EM] ln(likehood) = -721.6311530788772
    [##############......][GMM-EM] ln(likehood) = -721.6307378415466
    [##############......][GMM-EM] ln(likehood) = -721.6303217979058
    [##############......][GMM-EM] ln(likehood) = -721.6299049486149
    [##############......][GMM-EM] ln(likehood) = -721.6294872943191
    [###############.....][GMM-EM] ln(likehood) = -721.629068835655
    [###############.....][GMM-EM] ln(likehood) = -721.6286495732446
    [###############.....][GMM-EM] ln(likehood) = -721.6282295076962
    [###############.....][GMM-EM] ln(likehood) = -721.6278086396065
    [###############.....][GMM-EM] ln(likehood) = -721.6273869695601
    [################....][GMM-EM] ln(likehood) = -721.626964498126
    [################....][GMM-EM] ln(likehood) = -721.6265412258616
    [################....][GMM-EM] ln(likehood) = -721.62611715331
    [################....][GMM-EM] ln(likehood) = -721.6256922810012
    [################....][GMM-EM] ln(likehood) = -721.6252666094513
    [#################...][GMM-EM] ln(likehood) = -721.6248401391639
    [#################...][GMM-EM] ln(likehood) = -721.6244128706305
    [#################...][GMM-EM] ln(likehood) = -721.6239848043234
    [#################...][GMM-EM] ln(likehood) = -721.6235559407073
    [#################...][GMM-EM] ln(likehood) = -721.6231262802306
    [##################..][GMM-EM] ln(likehood) = -721.6226958233284
    [##################..][GMM-EM] ln(likehood) = -721.62226457042
    [##################..][GMM-EM] ln(likehood) = -721.6218325219149
    [##################..][GMM-EM] ln(likehood) = -721.6213996782053
    [##################..][GMM-EM] ln(likehood) = -721.6209660396704
    [###################.][GMM-EM] ln(likehood) = -721.6205316066757
    [###################.][GMM-EM] ln(likehood) = -721.6200963795732
    [###################.][GMM-EM] ln(likehood) = -721.6196603586993
    [###################.][GMM-EM] ln(likehood) = -721.6192235443764
    [###################.][GMM-EM] ln(likehood) = -721.6187859369128
    [####################][GMM-EM] ln(likehood) = -721.6183475366026
    [####################][GMM-EM] ln(likehood) = -721.6183475366026
    ===== hit max turn =====
    [0, 0, 0, 0, 0, 0, 50]
\end{verbatim}

\section{附录:源代码(带注释)}

\begin{verbatim}
    import numpy as np
    import matplotlib.pyplot as plt
    from scipy.stats import multivariate_normal


    def gauss_vector(mu_list: float, sigma_list: float, dim: int = 2) -> list:
        """生成一个向量,其各个维度服从正态分布

        Args:
            mu_list:    正态分布均值,其中 mu_list[i] 是第 i 维的均值
            sigma_list: 正态分布方差,其中 sigma_list[i] 是第 i 维的方差
            dim:        向量维数

        Returns:
            返回向量
        """

        vector = []
        for i in range(0, dim):
            vector.append(np.random.normal(mu_list[i], sigma_list[i], 1)[0])

        return vector


    def vectors2xylist(vectors: list) -> tuple:
        """转换向量列表格式
        将
        [
            [x0, y0],
            [x1, y1],
            ...,
            [xn, yn],
        ]
        格式的 vectors 转换为
        [x0, x1, ... xn], [y0, y1, ... yn]
        两个列表

        Args:
            vectors:  待转换向量列表

        Returns:
            返回向量列表中各组向量的 x 坐标和 y 坐标
        """

        xs = []
        ys = []
        for vec in vectors:
            xs.append(vec[0])
            ys.append(vec[1])

        return xs, ys


    def k_means(train_data: list, cluster_count: int, dim: int = 2) -> tuple:
        """使用 k-means 算法进行分簇

        Args:
            train_data:    原始未分簇的数据列表
            cluster_count: 需要分几个簇
            dim:           每个数据的维度

        Returns:
            (分好的簇列表, 各个簇的中心点列表)
        """

        # 各个簇的中心点列表
        avg_list = []
        for i in range(0, cluster_count):
            avg_list.append(np.random.random((dim)))

        # 各个簇
        cluster_list = [train_data]
        for i in range(1, cluster_count):
            cluster_list.append([])

        limit = 1e-5
        while True:
            # 重新分簇
            cluster_list_new = []
            for i in range(0, cluster_count):
                cluster_list_new.append([])

            for cluster in cluster_list:
                for vec in cluster:
                    # 到各个簇中心点的距离
                    distance_list = []
                    for i in range(0, cluster_count):
                        distance_list.append(np.linalg.norm(np.array(vec) - avg_list[i]))
                    for i in range(0, cluster_count):
                        if distance_list[i] == min(distance_list):
                            cluster_list_new[i].append(vec)
            cluster_list = cluster_list_new

            # 重新计算簇的中心点
            avg_list_new = []
            for i in range(0, cluster_count):
                avg_list_new.append(np.mean(np.array(cluster_list[i]), axis=0))

            # 当所有中心点变化都不大时,判断为收敛
            finished_cluster = 0  # 变化不大的中心点个数
            for i in range(0, cluster_count):
                if np.linalg.norm(avg_list_new[i] - avg_list[i]) < limit:
                    finished_cluster += 1
            if finished_cluster != cluster_count:
                avg_list = avg_list_new
            else:
                break

        return cluster_list, avg_list


    def e_step(
        cluster_count: int,
        data_count: int,
        data: list,
        mu_list: list,
        sigma_list: list,
        pi_list: list,
    ) -> np.matrix:
        """GMM-EM 算法的 E-步骤:求 gamma_z 矩阵

        Args:
            cluster_count: 簇的个数
            data_count:    数据的个数
            data:          数据集
            mu_list:       各个簇高斯分布的均值列表
            sigma_list:    各个簇高斯分布的协方差矩阵列表
            pi_list:       各个簇高斯分布的权重列表

        Returns:
            返回 gamma_z 矩阵
        """

        gamma_z = np.zeros((data_count, cluster_count))
        for i in range(0, data_count):
            pi_gauss_sum = 0
            pi_gauss = np.zeros(cluster_count)
            for k in range(0, cluster_count):
                pi_gauss[k] = pi_list[k] * multivariate_normal.pdf(
                    data[i], mu_list[k], sigma_list[k]
                )
                pi_gauss_sum += pi_gauss[k]
            for k in range(0, cluster_count):
                gamma_z[i][k] = pi_gauss[k] / pi_gauss_sum

        return gamma_z


    def m_step(
        cluster_count: int,
        data_count: int,
        data: list,
        mu_list: list,
        sigma_list: list,
        pi_list: list,
        gamma_z: np.matrix,
        dim: int,
    ) -> tuple:
        """GMM-EM 算法的 M-步骤:更新 mu_list, sigma_list, pi_list

        Args:
            cluster_count: 簇的个数
            data_count:    数据的个数
            data:          数据集
            mu_list:       各个簇高斯分布的均值列表
            sigma_list:    各个簇高斯分布的协方差矩阵列表
            pi_list:       各个簇高斯分布的权重列表
            gamma_z:       gamma_z 矩阵
            dim:           数据的维度

        Returns:
            返回更新的 (mu_list, sigma_list, pi_list)
        """

        for k in range(0, cluster_count):
            # gamma_z 矩阵第 k 列的和
            sum_gamma_z = np.sum(gamma_z[:, k])

            # 更新 mu
            sum_gamma_z_x = np.zeros(dim)
            for i in range(0, data_count):
                sum_gamma_z_x += gamma_z[i][k] * data[i]
            mu_list[k] = sum_gamma_z_x / sum_gamma_z

            # 更新 sigma
            sum_gamma_z_x_mu = np.zeros((dim, dim))
            for i in range(0, data_count):
                sum_gamma_z_x_mu += gamma_z[i][k] * np.dot(
                    data[i] - mu_list[k], np.transpose(data[i] - mu_list[k])
                )

            # 更新 pi
            pi_list[k] = sum_gamma_z / data_count

        return mu_list, sigma_list, pi_list


    def gmm_em(
        train_data: list, train_data_count: int, cluster_count: int, dim: int = 2
    ) -> tuple:
        """使用 GMM-EM 算法进行分簇

        Args:
            train_data:       原始未分簇的数据列表
            train_data_count: 数据数量
            cluster_count:    需要分几个簇
            dim:              每个数据的维度

        Returns:
            (各个簇的高斯分布的均值, 各个簇的高斯分布的协方差, 各个簇的高斯分布的权重)
        """

        # 初始化参数
        mu_list = []
        sigma_list = []
        pi_list = []
        for i in range(0, cluster_count):
            mu_list.append(np.random.random((dim)))
            sigma_list.append(np.diag([1] * dim))
            pi_list.append(1 / cluster_count)

        ln_likehood_old = -np.inf
        turn = 0
        max_turn = 100  # 最大迭代次数
        while True:
            turn += 1

            # E 步骤
            gamma_z = e_step(
                cluster_count,
                train_data_count,
                train_data,
                mu_list,
                sigma_list,
                pi_list,
            )

            # M 步骤
            mu_list, sigma_list, pi_list = m_step(
                cluster_count,
                train_data_count,
                train_data,
                mu_list,
                sigma_list,
                pi_list,
                gamma_z,
                dim,
            )

            # 似然函数判断收敛
            ln_likehood = 0
            for i in range(0, train_data_count):
                px = 0
                for k in range(0, cluster_count):
                    px += pi_list[k] * multivariate_normal.pdf(
                        train_data[i], mu_list[k], sigma_list[k]
                    )
                ln_likehood += np.log(px)
            print("\r[", end="")
            print("#" * int(20 * turn / max_turn), end="")
            print("." * (20 - int(20 * turn / max_turn)), end="")
            print("]", end="")
            print("[GMM-EM] ln(likehood) =", str(ln_likehood), end="")
            if ln_likehood - ln_likehood_old < 1e-5 or turn >= max_turn:
                print("\r[" + "#" * 20 + "]", end="")
                print("[GMM-EM] ln(likehood) =", str(ln_likehood))
                if turn >= max_turn:
                    print("===== hit max turn =====")
                break
            else:
                ln_likehood_old = ln_likehood

        return mu_list, sigma_list, pi_list


    def gmm_em_do_cluster(
        cluster_count: int,
        data_count: int,
        data: list,
        mu_list: list,
        sigma_list: list,
        pi_list: list,
    ) -> list:
        """对 GMM-EM 算法处理后的数据进行聚类

        Args:
            cluster_count: 簇的个数
            data_count:    数据的个数
            data:          数据集
            mu_list:       各个簇高斯分布的均值列表
            sigma_list:    各个簇高斯分布的协方差矩阵列表
            pi_list:       各个簇高斯分布的权重列表

        Returns:
            返回聚类列表
        """

        gamma_z = e_step(
            cluster_count,
            data_count,
            data,
            mu_list,
            sigma_list,
            pi_list,
        )

        cluster_list = []
        for i in range(0, cluster_count):
            cluster_list.append([])
        for i in range(0, data_count):
            cluster_index = list(gamma_z[i]).index(max(gamma_z[i]))
            cluster_list[cluster_index].append(data[i])

        return cluster_list


    def gmm_em_test(
        cluster_count: int,
        test_data_count: int,
        mu_list: list,
        sigma_list: list,
        pi_list: list,
        test_mu: list,
        test_sigma: list,
    ) -> list:
        """对 GMM-EM 算法聚类的效果进行测试

        Args:
            cluster_count:   簇的个数
            test_data_count: 测试数据的个数
            mu_list:         各个簇高斯分布的均值列表
            sigma_list:      各个簇高斯分布的协方差矩阵列表
            pi_list:         各个簇高斯分布的权重列表
            test_mu:         生成测试数据的均值
            test_sigma:      生成测试数据的协方差矩阵

        Returns:
            返回聚类后,各个簇的元素个数
        """

        test_cluster = []
        for i in range(0, test_data_count):
            test_cluster.append(np.random.multivariate_normal(test_mu, test_sigma))
        test_cluster_list = gmm_em_do_cluster(
            cluster_count,
            test_data_count,
            test_cluster,
            mu_list,
            sigma_list,
            pi_list,
        )

        return [len(i) for i in test_cluster_list]


    def draw_clusters(
        title: str, cluster_list: list, avg_list: list, new_figure: bool = True
    ) -> None:
        """绘制聚类散点图

        Args:
            title:        表标题
            cluster_list: 聚类列表
            avg_list:     各聚类的中心点列表
            new_figure:   是否需要新建一张图
        """

        if new_figure:
            plt.figure()
        plt.title(title)
        for i in range(0, len(cluster_list)):
            xs, ys = vectors2xylist(cluster_list[i])
            plt.scatter(xs, ys)
            plt.scatter(avg_list[i][0], avg_list[i][1], c="b", s=100)


    def load_data(filename: str, dim: int) -> list:
        """从文件中加载数据
        文件格式必须为
        x0,x1,...,xn,c0
        ...,
        x0,x1,...,xn,cm
        其中每一行的前 dim 个数据为训练数据向量的各个维度,最后一个数据为此行数据的类别

        Args:
            filename: 文件名
            dim:      数据维度

        Returns:
            返回加载的数据,格式为
            [
                [[x0, x1, ..., xn], c0],
                ...,
                [[x0, x1, ..., xn], c0],
            ]
        """

        result = []
        with open(filename, "r") as f:
            for line in f:
                data_line = [float(i) for i in line.removesuffix("\n").split(",")]
                vec = data_line[:-1]
                class_id = int(data_line[-1])
                result.append([vec, class_id])

        return result


    def uci() -> None:
        """运行 UCI 数据集"""

        uci_data_dim = 4
        uci_cluster_count = 7
        uci_train_data_raw = load_data("lab3/iris.data", uci_data_dim)
        uci_test_data_raw = load_data("lab3/iris_test.data", uci_data_dim)

        uci_train_data = [np.array(i[0]) for i in uci_train_data_raw]
        uci_train_data_count = len(uci_train_data)
        mu_list, sigma_list, pi_list = gmm_em(
            uci_train_data, uci_train_data_count, uci_cluster_count, uci_data_dim
        )

        uci_test_data = [np.array(i[0]) for i in uci_test_data_raw]
        uci_test_data_count = len(uci_test_data)
        cluster_list = gmm_em_do_cluster(
            uci_cluster_count,
            uci_test_data_count,
            uci_test_data,
            mu_list,
            sigma_list,
            pi_list,
        )
        print([len(i) for i in cluster_list])


    def custom() -> None:
        """运行自定数据集"""

        # ===== 初始参数 =====
        dim = 2  # 数据维度
        cluster_count = 3  # 簇的个数

        # ===== 训练数据 =====
        train_data = []
        train_data_count_per_cluster = 100  # 每个簇中的数据个数
        train_data_count = cluster_count * train_data_count_per_cluster  # 训练数据总个数
        for i in range(0, train_data_count_per_cluster):
            train_data.append(np.random.multivariate_normal([4, 4], [[4, 1], [1, 4]]))
            train_data.append(np.random.multivariate_normal([10, 10], [[1, 0], [0, 1]]))
            train_data.append(np.random.multivariate_normal([1, -7], [[3, 2], [2, 3]]))

        # ===== k-means =====
        cluster_list, avg_list = k_means(train_data, cluster_count, 2)
        draw_clusters("k-means", cluster_list, avg_list)

        # ===== GMM-EM =====
        mu_list, sigma_list, pi_list = gmm_em(
            train_data, train_data_count, cluster_count, dim
        )
        cluster_list = gmm_em_do_cluster(
            cluster_count, train_data_count, train_data, mu_list, sigma_list, pi_list
        )
        draw_clusters("GMM-EM", cluster_list, mu_list)

        # ===== 测试数据 =====
        test_data_count = 500
        print(
            "GMM-EM test result:",
            gmm_em_test(
                cluster_count,
                test_data_count,
                mu_list,
                sigma_list,
                pi_list,
                test_mu=[4, 4],
                test_sigma=[[4, 1], [1, 4]],
            ),
        )

        # ===== 绘图 =====
        plt.show()


    if __name__ == "__main__":
        use_uci = False

        if use_uci:
            # ===== UCI 数据集 =====
            uci()
        else:
            # ===== 自定义数据集 =====
            custom()

\end{verbatim}
