\section{设计思想(本程序中的用到的主要算法及数据结构)}

% 多元高斯分布公式
\def\gauss{\mathcal{N}\left(\mathbf{x}_i|\mu_k,\Sigma_k\right)}

% 从 1 到 K 求和
\def\sumk{\sum\limits^K_{k=1}}

% 从 1 到 n 求和
\def\sumn{\sum\limits^n_{i=1}}

% gamma z
\def\gammaz#1{\gamma\left(\mathbf{z}_{i {#1}}\right)}


\subsection{k-means}

我们有$n$个训练样本
\begin{equation}
    \mathbf{X}=
    \begin{bmatrix}
        \mathbf{x}_1 \\ \mathbf{x}_2 \\ \vdots \\ \mathbf{x}_n
    \end{bmatrix}
\end{equation}
其中样本$\mathbf{x}_i$是一个$d$维向量。

目标是要将这些样本划分到$K$个簇$C_1, C_2, \ldots C_K$中,同一簇内部的点比较“紧密”。想要衡量各个簇内部的点的“紧密”程度,可以用簇内各点到簇中心点$\mu$的距离平方和来表示,即
\begin{equation}
    E=\sumk\sum_{\mathbf{x}\in C_k}|\mathbf{x}-\mu_k|^2
\end{equation}
其中$E$越小,“紧密”程度越高。所以只需要最小化$E$。

使用迭代完成这个过程:
\begin{itemize}
    \item 初始化$K$个簇的中心点$\mu$,可以随机取值
    \item 计算所有点到这$K$个中心点的距离,若点$\mathbf{x}_i$到$\mu_k$的距离最近,则将点$\mathbf{x}_i$归入簇$C_k$
    \item 根据新的簇划分,重新计算各个簇的中心点$\mu_k$。若新的中心点与旧的中心点之差小于给定值,则认为已经收敛;否则重新回到第二步
\end{itemize}

\subsection{高斯混合分布}

\subsubsection{样本生成}

我们有$n$个训练样本
\begin{equation}
    \mathbf{X}=
    \begin{bmatrix}
        \mathbf{x}_1 \\ \mathbf{x}_2 \\ \vdots \\ \mathbf{x}_n
    \end{bmatrix}
\end{equation}
其中样本$\mathbf{x}_i$是一个$d$维向量,由一个$d$元高斯分布生成,即
\begin{equation}
    \mathcal{N}\left(\mathbf{x_i}|\mu,\Sigma\right)=
    \dfrac{1}{\left(2\pi\right)^{\frac{d}{2}}|\Sigma|^{\frac{1}{2}}}\exp\left(-\dfrac{1}{2}\left(\mathbf{x_i}-\mu\right)^T\Sigma^{-1}\left(\mathbf{x_i}-\mu\right)\right)
\end{equation}

\subsubsection{用于高斯混合模型GMM的EM}

GMM定义为
\begin{equation}
    P\left(\mathbf{x}\right)=\sumk\pi_k\mathcal{N}\left(\mathbf{x}|\mu_k,\Sigma_k\right)
\end{equation}
其可以看作是由$K$个$d$元高斯分布混合而成的,其中$\pi_k$为相应的高斯分布的权重,满足
\begin{equation}
    \label{pik}
    \sumk\pi_k=1
\end{equation}

\subsubsection{聚类}

在生成数据的时候,可以将$\mathbf{x_i}$看作是从$K$个混合的高斯分布中挑出一个单独的高斯分布所生成的。所以可以设有$K$维向量$\mathbf{z}$,表示对$K$个高斯分布的选择。

$\mathbf{z}$只有一个元素为$1$,其余全为$0$,且
\begin{equation}
    P\left(\mathbf{z_k}=1\right)=\pi_k
\end{equation}
则$\mathbf{x}_i$属于簇$k$的概率为
\begin{align}
    P\left(\mathbf{z}_k=1|\mathbf{x_i}\right)
    &= \dfrac{P\left(\mathbf{z}_k=1\right)P\left(\mathbf{x}_i|\mathbf{z}_k=1\right)}{P\left(\mathbf{x}_i\right)} \\
    &= \dfrac{P\left(\mathbf{z}_k=1\right)P\left(\mathbf{x}_i|\mathbf{z}_k=1\right)}{\sumk P\left(\mathbf{z}_k=1\right)P\left(\mathbf{x}_i|\mathbf{z}_k=1\right)} \\
    &= \dfrac{\pi_k\gauss}{\sumk\pi_k\gauss}
\end{align}
因此对于一个新样本$\mathbf{x}_i$,欲将其归入簇$1, 2, \ldots, K$号中的一个,只需要找到能令$P\left(\mathbf{z}_k=1|\mathbf{x}_i\right)$最大的$k$,就找到了最有可能生成$\mathbf{x}_i$的高斯分布$\gauss$,也就得到了$\mathbf{x}_i$的簇号$k$。

上述过程可以表示为
\begin{align}
    k
    &= \arg\max_k P\left(\mathbf{z}_k=1|\mathbf{x}_i\right) \\
    \label{likehood}
    &= \arg\max_k \dfrac{\pi_k\gauss}{\sumk\pi_k\gauss}
\end{align}
则得到的$k$就是$\mathbf{x}_i$的簇号。

从公式\ref{likehood}可以看出,求出参数$\pi_k, \mu_k, \Sigma_k$是GMM聚类的关键($k\in \{1, 2, \ldots K\}$)。可以用最大化似然函数的对数来完成此过程,即
\begin{align}
    \mathcal{L}\left(\mathbf{X}\right)
    &= \ln P\left(\mathbf{X}|\pi, \mu, \Sigma\right) \\
    &= \ln\prod^n_{i=1}\left(\sumk\pi_k\gauss\right) \\
    \label{lnlikehood}
    &= \sumn\ln\left(\sumk\pi_k\gauss\right)
\end{align}

对其关于$\mu_k$求导,得
\begin{align}
    \dfrac{\partial\mathcal{L}\left(\mathbf{X}\right)}{\partial\mu_k}
    &= \sumn\dfrac{\pi_k\dfrac{\partial\gauss}{\partial\mu_k}}{\sumk\pi_k\gauss} \\
    \label{dmu}
    &= \sumn\dfrac{-\pi_k\gauss\Sigma^{-1}_k\left(\mathbf{x}_i-\mu_k\right)}{\sumk\pi_k\gauss}
\end{align}
(多元正态分布对$\mu$的求导过程见附录)

由于$\mu_k$同时位于指数和系数,故无法获得导数为$0$的解析解,但是可以用迭代的方式逐渐逼近$\mu_k$。为了方便后续化简,设
\begin{equation}
    \label{gamma}
    \gammaz{k}=\dfrac{\pi_k\gauss}{\sumk\pi_k\gauss}
\end{equation}

令式\ref{dmu}这个求导结果为$0$,等式两侧同乘$\Sigma$,可得
\begin{equation}
    \sumn\gammaz{k}\left(\mathbf{x}_i-\mu_k\right)=0
\end{equation}
化简得
\begin{equation}
    \label{mu}
    \mu_k=\dfrac{\sumn\left(\gammaz{k}\mathbf{x}_i\right)}{\sumn\gammaz{k}}
\end{equation}

同理,令式\ref{lnlikehood}关于$\Sigma_k$的导数为$0$,可得
\begin{equation}
    \label{Sigma}
    \Sigma_k=\dfrac{\sumn\left(\gammaz{k}\left(\mathbf{x}_i-\mu_k\right)\left(\mathbf{x}_i-\mu_k\right)^T\right)}{\sumn\gammaz{k}}
\end{equation}
(多元正态分布对$\Sigma$的求导过程见文献\cite{PRML})

最后只需要求出$\pi_k$的迭代式。考虑式\ref{pik},可以使用拉格朗日乘数法,尝试最大化
\begin{equation}
    \mathcal{L}\left(\mathbf{X}\right)+\lambda\left(\sumk\pi_k-1\right)
\end{equation}
即
\begin{equation}
    \sumn\ln\left(\sumk\pi_k\gauss\right)+\lambda\left(\sumk\pi_k-1\right)
\end{equation}
对其中的$\pi_k$求导,并令导数为$0$,得
\begin{equation}
    \label{dpi}
    \sumn\dfrac{\gauss}{\sumk\pi_k\gauss}+\lambda=0
\end{equation}
对式\ref{dpi}两侧同乘$\pi_k$,得
\begin{equation}
    \label{pis}
    \pi_k\sumn\dfrac{\gauss}{\sumk\pi_k\gauss}+\pi_k\lambda=0
\end{equation}
对$k\in\{1,2,\ldots K\}$,可以像这样构造$K$个等式,将这$K$个等式相加,可得
\begin{align}
    \sum^K_{k=1}\left(\pi_k\sumn\dfrac{\gauss}{\sumk\pi_k\gauss}\right)+\sum^K_{k=1}\pi_k\lambda &= 0 \\
    \sum^K_{k=1}\left(\sumn\dfrac{\pi_k\gauss}{\sumk\pi_k\gauss}\right)+\sum^K_{k=1}\pi_k\lambda &= 0 \\
    \sumn\left(\sum^K_{k=1}\dfrac{\pi_k\gauss}{\sumk\pi_k\gauss}\right)+\sum^K_{k=1}\pi_k\lambda &= 0 \\
    \sumn\left(\dfrac{\sum\limits^K_{k=1}\pi_k\gauss}{\sumk\pi_k\gauss}\right)+\sum^K_{k=1}\pi_k\lambda &= 0 \\
    \sumn1+\lambda &= 0 \\
    \lambda &= -n
\end{align}
注意到式\ref{pis}也可用式\ref{gamma}化简为
\begin{equation}
    \sumn\gammaz{k}+\pi_k\lambda=0
\end{equation}
再利用$\lambda=-n$,可得
\begin{equation}
    \label{pi}
    \pi_k=\dfrac{1}{n}\sumn\gammaz{k}
\end{equation}

至此,三个重要参数$\pi_k, \mu_k, \Sigma_k$的迭代公式分别由式\ref{mu}、\ref{Sigma}、\ref{pi}给出。

EM算法可以描述为:
\begin{itemize}
    \item 对于每一个簇$k$,都初始化对应的$\mu_k$、$\Sigma_k$和$\pi_k$
    \item E步骤:对于每一个训练样本$\mathbf{x}_i$,都计算$\gammaz{k}$,即
    \begin{equation*}
        \gammaz{k}=\dfrac{\pi_k\gauss}{\sumk\pi_k\gauss}
    \end{equation*}
    \item M步骤:利用$\gammaz{k}$重新估计参数$\pi_k$、$\mu_k$和$\Sigma_k$
    \begin{align*}
        \mu_k    &= \dfrac{\sumn\left(\gammaz{k}\mathbf{x}_i\right)}{\sumn\gammaz{k}} \\
        \Sigma_k &= \dfrac{\sumn\left(\gammaz{k}\left(\mathbf{x}_i-\mu_k\right)\left(\mathbf{x}_i-\mu_k\right)^T\right)}{\sumn\gammaz{k}} \\
        \pi_k    &= \dfrac{1}{n}\sumn\gammaz{k}
    \end{align*}
    \item 当对数似然函数
    \begin{equation*}
        \mathcal{L}\left(\mathbf{X}\right)=\sumn\ln\left(\sumk\pi_k\gauss\right)
    \end{equation*}
    的变化量或某个参数的变化量低于某个阈值时,可以认为EM算法收敛\cite{PRML};否则重新进行E和M步骤
\end{itemize}
