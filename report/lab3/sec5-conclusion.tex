\section{结论}

由于k-means假设数据呈球状分布(欧氏距离),而GMM使用多元高斯分布做更加普遍的假设。当样本各个维度间的相关系数不为0时,k-means的假设不成立,所以聚类的准确率可能会所下降。从另一个角度看,k-means可以看作是假设协方差矩阵是对角阵。

k-means和GMM-EM算法的性能都较好,可以比较快地收敛。

GMM-EM算法有一个严重的问题:当某一个分量的标准差$\sigma\rightarrow 0$时,这各分量对应的高斯分布将趋于$\infty$,因此对数似然也会趋于$\infty$,这会使得GMM出现过拟合问题。但是我们可以用贝叶斯方法(相当于在最大似然的基础上添加正则项)来解决这个问题,或者当检测到⾼斯分量收缩到⼀个点,那么就将它的均值重新设定为⼀个随机选择的值,并且重新将它的⽅差设置为某个较⼤的值,然后继续最优化\cite{PRML}。
